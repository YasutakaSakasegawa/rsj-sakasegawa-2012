%%% 論文(A)と一般記事用(B)のテンプレート
%%%   必要に応じてどちらかを使ってください.

%%% A. 論文
\documentclass[paper]{jrsj}
\usepackage{graphicx}

%% 頁(指定する必要なし)
%\setcounter{page}{1}
%% 通巻頁(指定する必要なし)
%\volpage{101}

%% 論文の種類を記入(学術・技術論文 / 学術論文 / 討論など)
\typeofpaper{学術・技術論文}
%% 発行年(コメントアウトしておく)
%\Year{2000}
%% 巻数(コメントアウトしておく)
%\Vol{18}
%% 号数(コメントアウトしておく)
%\No{1}
%% 邦文題名
\title{
国際教育のプロジェクトへのアジャイル開発方法論スクラムの適用とマネジメント(仮)
}
%% 邦文副題名(ある場合のみ)
\subtitle{}
%% 英文題名
\etitle{English Title}
%% 英文副題名(ある場合のみ)
\esubtitle{}
%% 著者のリスト
\authorlist{%
 \authorentry{酒瀬川 泰孝}{Yasutaka Yasutaka}{AIIT}
 \authorentry{川木 富美子}{Tomiko Kawaki}{AIIT}
 \authorentry{須澤 秀人}{Hideto Suzawa}{AIIT}
 \authorentry{木崎 悟}{Satoru Kizaki}{AIIT}
 \authorentry{土屋 陽介}{Yosuke Tsuchiya}{AIIT}
 \authorentry{中鉢 欣秀}{Yoshihide Chubachi}{AIIT}
}
%% 著者の所属
\affiliate[AIIT]{産業技術大学院大学}{Advanced Institute of Industrial Technology}

%% 原稿受付日(記述しない,空欄に)
\received{}

\begin{document}
% % 英文要旨
\begin{abstract}
Lorem ipsum dolor sit amet, consectetur adipiscing elit. Pellentesque sit amet nibh odio. Nam non bibendum velit. Vestibulum ante ipsum primis in faucibus orci luctus et ultrices posuere cubilia Curae; Ut sagittis elit ac sapien pulvinar pellentesque fermentum arcu aliquam. Integer sollicitudin lorem ac nulla mattis vehicula. Suspendisse sed mauris magna, a sagittis neque. Praesent egestas purus eu erat adipiscing in porta enim auctor. Vestibulum egestas neque in orci congue gravida. Pellentesque porttitor magna ac magna ornare pretium. Ut vestibulum sapien ut sem rutrum et suscipit velit consequat. Ut cursus libero et nisi faucibus pretium luctus erat sodales. Morbi lacus.
\end{abstract}
%% 英文キーワード
\begin{keywords}
Lorem ipsum dolor sit amet
\end{keywords}
\maketitle
\small
%% 本文
\section{はじめに}

本論文では,ロボット開発プロジェクトのプロジェクトマネージャや技術者の教育を育成する国際PBL(Project
Based Learning)にAgile開発のプロジェクトマネジメント方法論の一つであるスクラムを導入し,ベトナム国家大学と共同で国際教育プロジェクトを実施した.プロジェクトマネジメントにアジャイル開発方法論スクラムを導入した結果について考察加え報告する.

\subsection{背景}

産業のグローバル化が進み,ビジネスライフサイクルの超短期化,ビジネスピードがその企業のマーケットシェア決めている,ロボット開発においても, M2Mの様にロボットを情報システムに融合し,新しいサービスとしてビジネスを展開する事例も増えつつある.

このようなロボットに対する考え方の変化の中では,ロボット開発プロジェクトに対する開発期間の短期化と要求事項の変化へ対応した開発方法論が求められる.

そのため,市場環境の変化の激化の下で開発プロジェクトの成功に貢献する人材育成の成否が重要な課題となっている、特に開発プロジェクトを効果的にマネジメントできるプロジェクトマネージャの育成が課題である.

\subsection{PBFの目的}
このため,2012年度のPBLにおいて,迅速かつ軽量で,
要求の変化に柔軟に対応できる開発手法であるAgile開発方法論の代表的な手法であるScrumに着目し,
Scrum開発手法を適用したロボット制御プログラム開発の教育プロジェクトをベトナム国家大学ハノイ校と実施した.

この国際PBLでは,2国間でのAgile開発方法論Scrumを適用したソフトウェア開発プロジェクトを通して,
以下の能力を獲得した人材の育成を目指した.

\begin{enumerate}
  \item 国際環境におけるロボット開発プロジェクトをマネジメントできる人材の育成,特に開発における技術者やマネジメントの責任や役割,振る舞いを理解・習得する
  \item 異文化,異なる言語,多様な背景を持つ学生が同じPBLを実践し議論することで,多様な発想を持った人材の育成
  \item 上記の教育効果を実現するために必要な教材やメソッドの作成,これらを用いた教育の実施と検証を通してプロジェクトチームへの教育能力の獲得.  
\end{enumerate}


\subsection{PBLの題材}

題材は,ロボットとインターネット,スマートフォンの融合をテーマに,RSNPプロトコルを利用しインターネット介して,
Android端末とLego Mindstormsを利用したロボットを操作するアプリケーションを開発した
.PBLに,ロボットを使う狙いは以下の通りである

\begin{enumerate}
  \item 開発の課程・プロジェクト実践成果を可視化できる
  \item 実際に動くものを一緒に作ることで言葉の壁を越えた深い理解を得ることができる.  
\end{enumerate}

\section{アジャイル開発方法論}
\subsection{アジャイル開発方法論とは}
アジャイル開発方法論とは、要件の変化に対応しソフトウェアが「柔軟で変更に強く」なるように
開発プロセスがデザインされた開発方法論の総称である,2000年以後,ソフトウェア開発の分野で注目されている.

代表的な手法は,下記がある.

\subsection{アジャイル開発の歴史}
アジャイル開発方法論は199年代前半に生み出された,ソフトウェアの開発方法論である.

\begin{enumerate}
  \item XP Exterime Programming
  \item Scrum
  \item Lean Software Development   
\end{enumerate}

他にも様々な手法が提案されているが,いずれも以下の共通の特徴を有している.

\begin{itemize}
  \item ソフトウェアを小さな機能単位に分割し,開発優先順を定め,これに基づき漸進的に繰り返し開発する.
  \item 開発プロジェクト内のそれぞれのチームは5~10名程度のメンバーで編成され,自律的にかつそれぞれのチームメンバーが設計~実装~テストまでの一連の開発を実施すること可能な多能工から成る.これを「self-organizing, cross-functional teams」と呼ぶ.
  \item ウォーターフォール型開発方法論やその他の逐次型開発方法論と比べ,軽量な開発方法論である.最小限のプロセスとプラクティス(効果的とされる経験則)から構成される.   
\end{itemize}

【Fig1.ソフトウェア開発方法論の変遷】

図に示すとおり,1970年代ウォータフォール型の開発方法論が普及していった,これは現在も利用されている.

ウォーターフォール・モデルの原型となったのは、1970年に発表されたウィンストン・W・ロイス(Winston W.Royce)の
論文[6]だとされる。
しかしこの論文では「ウォーターフォール」という語も使われていない。
実は,この論文は、ウォーターフォール・モデルの危険性を指摘しており,文書化、レビューの重要性ともに、
下流工程を予備的に行って上流工程へ戻る「フィードバックループ」が提唱されている. 
Royceは、リスクを回避するために5 つStepを提言しており,その3 番目が“DO IT TWICE”つまり,
開発プロセスの反復であった. 
1980年代に入り,Royceの息子であるWalker Royceは反復型のソフトウェア開発プロセスの代表ともいえる
RUP (Rational Unified Process,1980) を提唱した.
1990年代に入り綿密な計画と,厳格な統制、管理に基づく「重量級の開発手法」に対して,
迅速さと要求の変化への柔軟な対応をうたった「軽量ソフトウェア開発手法」が考案された,
アジャイル開発方法論の始まりである.
. 2001年に「軽量ソフトウェア開発手法」において名声のある人々が会し、
「アジャイルソフトウェア開発手法」と呼称を変えた。 
その後、一部は、非営利組織 Agile Alliance を設立し、アジャイル開発を推進する活動を行っている。

\subsection{アジャイルマニフェストと価値}
様々なアジャイル開発プロセスの提唱者たちが,
アジャイル開発プロセスは共通的な価値観に基づいていると気づき,
共同で考えた価値と原則がある.

【Fig2アジャイルマニフェスト[8]】
 
http://agilemanifesto.org/principles.html

これは,アジャイルマニフェストと呼ばれ,アジャイル開発プロセスにおける共通の思想や価値観を表している.

\begin{enumerate}
  \item プロセスやツールよりも個人と対話を,
  \item 包括的なドキュメントよりも動くソフトウェアを,
  \item 契約交渉よりも顧客との協調を,
  \item 計画に従うことよりも変化への対応を,    
\end{enumerate}

価値とする.

すなわち,左記のことがらに価値があること認めながらも,私たちは右記のことがらにより価値をおく.


 2.4アジャイルマニフェストと原則
アジャイルマニフェストでは12の原則も表明している.
1) 顧客満足を最優先し,価値のあるソフトウェアを早く継続的に提供します.
2) 要求の変更はたとえ開発の後期であっても歓迎します.変化を味方につけることによって,お客様の競争力を引き上げます.
3) 動くソフトウェアを,2-3週間から2-3ヶ月というできるだけ短い時間間隔でリリースします.
4) ビジネス側の人と開発者は,プロジェクトを通して日々一緒に働かなければなりません.
5) 意欲に満ちた人々を集めてプロジェクトを構成します.
環境と支援を与え仕事が無事終わるまで彼らを信頼します.
6) 情報を伝えるもっとも効率的で効果的な方法はフェイス・トゥ・フェイスで話をすることです.
7) 動くソフトウェアこそが進捗の最も重要な尺度です.
8) アジャイル・プロセスは持続可能な開発を促進します.
9) 一定のペースを継続的に維持できるようにしなければなりません.
10) 技術的卓越性と優れた設計に対する不断の注意が機敏さを高めます.
シンプルさ(ムダなく作れる量を最大限にすること)が本質です.
11) 最良のアーキテクチャ・要求・設計は,自己組織的なチームから生み出されます.
12) チームがもっと効率を高めることができるかを定期的に振り返りそれに基づいて自分たちのやり方を最適に調整します.

 2.6アジャイル開発プロセスのプラクティス.
アジャイル開発プロセスでは,開発において良い効果をもたらすとされる慣行を「プラクティス」呼んでいる.プラクティスに,アジャイルマニフェストと同様に共通的な概念がある.特に主なプラクティスが「繰り返し型開発」(反復型開発ともいう)と「小ロット開発」である.
1)  繰り返し型開発
	特徴
	まず,短く,固定された開発期間を定める.これをイテレーションと呼ぶ
	イテレーションを繰り返しながら,漸進的にソフトウェアを開発していく
	イテレーション毎に顧客からフィードバックを得ることで,ソフトウェアのビジネス上の価値を最大化させる
	顧客からの要求やフィードバックに基づき,開発の要件を組み替え,計画を変更しつづける
2)	小ロット開発
	特徴
	小さく,顧客にとって意味のある単位にソフトウェアへの要望を分割して少しづつ開発する.
	イテレーション毎に要望の必要性を見直し,まだ開発していない要望と,現在での要望のフトウェアのビジネス上の価値を比較しフトウェアのビジネス上の価値が最大になるように計画を変更しつづける
3)	ソフトウェア開発の不確実性を解決
「繰り返し型開発」と「小ロット開発」は,ソフトウェア開発の不確実性を解決するため,少しずつ開発することでリスクを低減し, ソフトウェア開発の不確実性を解決を最大化させるである.

\subsection{ウォーターフォール型開発プロセスとアジャイル開発プロセスの違い}

\subsection{海外でのアジャイル開発方法論の普及}
Gartnerのアナリスト,Thomas Murphy と David Nortonらによれば, 2012年までに、
ソフトウェア開発プロジェクト全体の80%でアジャイル開発方法論が利用されると述べている.[2]

“agile development methods will be utilized in 80% of all software development projects”.

また,IPAの調査[3]によれば, Forrester2010調査[4]を引用し,” “2009年度の第3四半期に実施した調査で、約35%の開発プロジェクトで アジャイル型開発手法を採用しているという結果が示されている。”と報告している.またVersion One社の調査[5]を引用する形で, 回答者のうち2/3 以上の企業が、自社において3つ 以上のチームでアジャイルを実践しており,中でもScrum型の開発プロセスとその変化形が2/3を閉めていると報告している。

\subsection{日本におけるアジャイル開発方法論の普及}

IPAの調査[2]では,日本におけるアジャイル開発方法論の普及は欧米に比べて大変遅れていること,アジャイル開発方法論の代表的な手法であるスクラムの認定資格者数の推移から示している.

\subsection{なぜアジャイル開発プロセスか?}
リレー競争のように要求定義,設計,実装,テスト,顧客の受け入れ試験を順番に行うウォーターフォール型開発プロセスは,顧客の求める機能や仕組みが顧客の要求通りかどうかを確認できるタイミングが開発の終盤にあるため,手戻りが発生した場合のリスクが大きい.また,前工程の完了確認をもって,次の工程に入るため,各工程の遅れが,開発プロジェクト全体の遅延につながりやすく,結果として,最適な時期に最適な製品やサービスを市場へ投入することができなくなる

その結果,The standish Group,2006 Chaos Reportによれば,ソフトウェアの機能のうち64%が利用されないでいるという.

\section{アジャイル開発方法論スクラム}

\subsection{スクラムとは}
スクラムは日本の産業界のベストプラクティスに基づいたアジャイル開発プロセスの1つである. 
スクラムは.高いビジネス価値をより早期に顧客に,そして継続的に提供することを目的にしている.

スクラムはアジャイル開発プロセスの中でもプロセスが明示的に定義された開発プロセスで軽量(Simple)かつ理解しやすいという特徴がある.

スクラムは動作するソフトウェアをあらかじめ定めた周期で速やかに繰り返し確認していく、通常2週間から1ヶ月周期である.スクラム開発プロセスでは,顧客も開発に参加する.顧客は開発要件に優先順位をつける.これは,顧客のビジネス上の価値に基づく. チームは優先度の高い機能をから開発する.顧客に納める最良の開発方法はチーム自ら決定する.

顧客は,2週間~1か月ごとに動作するソフトウェアを確認することができる,そしてそのままリリースするか、別の繰り返しで機能拡張するかを決めることができる.

開発チームは, 2週間~1か月ごとに顧客から開発したソフトウェアについてフィードバックをえる,これにより,より深く顧客のビジネスや要求を理解することができる.その結果ソフトウェアの価値と品質を高めることが可能になる.

\subsection{スクラムの源典}
スクラムの源典は野中郁次郎の論文\cite{takeuchi1986new}である.

スクラム的手法を以前から開発プロジェクトで使っていた企業として、富士ゼロックス、キヤノン、本田技研工業、日本電気、セイコーエプソン、ブラザー工業、3M、ゼロックス、ヒューレット・パッカードなどがある。これらのプロジェクトについては、野中郁次郎と竹内弘高が Harvard Business Review 誌に "The New New Product Development Game" として発表している(1986年1-2月)。逆に言えば、この論文がスクラムという用語の元となった。

その中で野中らは「製品開発におけるリレー競走のアプローチは、最大のスピートと柔軟性というゴールとは競合するだろう。」と述べており,

これに応えるアイデアとして「代わりに、 全体的もしくはラグビーのようなアプローチ(チームは離れたゴールまでかたまりとして進もうとし、ボールを後ろへ前へパスする)は、今日の競争力のある要求の実現にとって、より役立つだろう」と述べ自己組織化されたチームによる反復型のアプローチを提唱した,これがアジャイル開発プロセスの一つスクラムの源典となった.

\subsection{スクラムの誕生}


スクラムが開発手法として登場したのは1993年、Jeff Sutherlandが野中らの論文をヒントにラウンドトリップ・エンジニアリング(一種の反復型開発)を取り入れたオブジェクト指向プログラミング設計・分析ツールを構築したのが最初とされている。

1995に Sutherland と Schwaber describeらが “Scrum Methodology”として発表,

2011  Schwaber とMike Beedle らが 著書“Agile Software Development with Scrum”にてそのプロセスを詳細に著述

2011  Scrum ver 1.2が発表された.

\subsection{スクラムの理論}
スクラムはアジャイル開発プロセスの中でプロセスが明示的に定義された開発プロセスであり,経験的プロセス制御の理論(経験主義)を基本にしている。経験主義とは、実際の経験と既知に基づく判断によって知識が獲得できるというものである。スクラムでは、反復的で漸進的な手法を用いて、予測可能性の最適化とリスクの管理を行う。[8]

経験的プロセス制御の実現は、透明性・検査・適応の3つの概念に支えられている。

1)	透明性

スクラムで重要な点は,成果物の責任持つものに対して,プロセスを見える化することである.例えば開発プロジェクトにおける用語が同じものを指す.作業完了の定義がプロジェクトのメンバー全員で共有されている.などである.

2)	検査

スクラムでは開発プロジェクトのゴールに進む際に起こりうる,好ましくない兆候を早く検知できるように,様式や成果物を頻繁に検査しなければならない.検査は作業の妨げになるほど頻繁に行うべきでは無い,検査の効果は,その分野で最も知識と経験がある優れた検査担当者が担当することで効果を発揮する.

3)	適応

もし,検査人が成果物の受け入れができないことを指摘した場合,開発プロセスやその構成物は直ちに見直されなければならない.それによりこれ以上の逸脱を防がなければならない.

スクラムでは,透明性と検査の効果を確保するため,4つの会議を設けている.これは後述する.

スプリント計画会議

デイリースクラム会議

スプリントレビュー会議

スプリントレトロスペクティブ会議

これらの理論は,OODAサイクルに非常によく似ている.OODAサイクルは,のことであり,これは

\subsection{スクラムの特徴}

スクラムには以下の特徴がある
1)	自己組織化されたチーム
2)	開発は,2週間から1ヶ月のスプリントと呼ばれる固定された期間を繰り返す中で行われる.
3)	顧客の要求はプロダクトバックログ呼ばれる優先順位付けされた要求事項リストで管理される.
4)	特定の技術的事項に寄らない(つまり.ソフト開発以外でも適用可能)
5)	アジャイル開発の環境を作り出し維持するため, アジャイル開発でよく知られた,いくつかの一般的なルールを適用する.
Product progresses in a series of month-long “Sprints”
Requirements are captured as items in a list of “Product Backlog”
No specific engineering practices prescribed
Uses generative rules to create an agile environment for delivering projects
One of the “agile processes”

\subsection{スクラムのマネジメントフレームワーク}
スクラムはシンプルなアジャイル開発のプロジェクトマネジメントのフレームワークである. 


There are 3 roles, 2 Artifacts,4 Ceremonies
Scrumは、以下の要素で構成されています。

ロール: プロダクトオーナー、スクラムマスター、開発者 
イベント: スプリント計画、デイリースクラム、スプリントレビュー、スプリントレトロスペクティブ 
成果物: プロダクトバックログ、スプリントバックログ、インペディメントリスト

\section{検証方法や解決方法の提案}


\section{プロジェクト概要}
目標:RNSPコンテストへ出場
フェーズ1:ベトナム国家大学の開発目担当メンバーのためのAgile開発フレームワークscrumの教材開発とオンサイトトレーニング
フェーズ2:ベトナム国家大学との国際分散アジャイル開発

\subsection{フェーズ1:Agile開発フレームワークscrumの教材開発とヴェトナム国家大学におけるオンサイトトレーニング}
VNU用教材を参考に書く.

\subsection{トレーニング(セミナー)の準備上の課題}
(7月)ベトナム国家大学側のPBL実施期間は6月下旬~8月末まで, つまり準備期間は約1カ月
トレーニングの準備自体をagile手法(Scrum)で行う
リユース可能教材の活用(カスタマイズ)
Scrum ブートキャンプ(アジャイル開発研修)へ参加
英語の講義台本を作成
リハーサル
完成!

\subsection{ベトナム国家大学用教材の作成}
トレーニング教材作成方針
•	Scrumの思想や理論を身につける
•	Scrumのドキュメントの作成が独力でできる
•	Scrumの行事が独力でできる
•	Scrumのスケジューリングが独力でできる
対応策
•	セミナーの中に演習を取り入れ,スムーズに本番に移行できるようにする
•	演習内容については詳細に文書化し,正しく演習が進行するようにする

\subsection{トレーニングの実施}
2012年7月に2日間,ベトナム国家大学ハノイ校で実施
紙飛行機プラクティスの実行の模様
プロダクトバックログの作成の模様
スプリントバックログの作成の模様
日次更新作業の模様


\subsection{フェーズ2:ベトナム国家大学との国際分散アジャイル開発}
プロジェクトの概要と体制と期間を書く

\subsection{開発の目的}
RSNPプロトコルを利用しインターネットを介して,Android端末とLego Mindstormsを利用したロボットを操作するアプリケーションを開発する

\subsection{プロジェクト概要}
AIIT(産業技術大学院大学)中鉢PBLおよびVNU(ベトナム国家大学ハノイ)PBLとのジョイントPBL
6月下旬~8月下旬までの約2カ月のプロジェクト

\subsection{開発体制}
日本側   : 3名 (PM経験者中心の社会人学生)
ベトナム側: 学部3年生4名と4年生6名  計10名
         (各種言語既習,チーム開発経験無)
お客様役は加藤教授

\subsection{開発するシステムの概要}
システムの要求仕様
•	Android端末を搭載したLEGO MINDSTORMSロボットを作成.
•	RSNPサーバにHTTP接続したブラウザから,ロボットの動作制御(前進,後退,右折,左折,停止)
•	RSNPサーバにHTTP接続したブラウザから,Android端末に内蔵されたカメラを操作.画像はサーバに転送され,サーバ側ブラウザで確認できる
•	RSNPサーバにHTTP接続したブラウザから,GPS情報(緯度,経度)を取得し,ブラウザに表示

技術:
C#,.NET,Microsoft Project, Microsoft Project Server

アーキテクチャ
クラウドシステム,AWS(amazonnクラウドサービス)へサーバーを配置

規模
約1K Step

特徴
・ アンドロイドOS用のロボット制御ソフトウェア
・ ベースとなるRSNPの機能はブラックボックス

\subsection{開発上の課題と制約条件}
超短期開発
遠隔地/開発期間FIX/機能は固定していない
高い品質を目指す

\subsection{ロボット制御ソフトの国際分散アジャイル開発のプロジェクトマネジメントのモデルを提示}
 プロジェクトに取り入れたアジャイル開発のプラクティスをプロジェクトマネジメントの9個のナレッジエリアに沿ってマトリックスに整理し、そのあと説明を文章で加える.
1	統合管理における工夫点
2.スコープ管理における工夫点
3.時間管理(スケジュール管理)における工夫点
4.コスト管理における工夫点
5.品質管理における工夫点
6.人的資源管理における工夫点
7.コミュニケーション管理における工夫点
8.リスク管理における工夫点
9.調達管理における工夫点

このプロジェクトで適用した開発プロセスの絵を入れる


\subsection{体制と役割の工夫 サービス指向のプロジェクトマネジメントと開発ゲーム}
1)AIIT側へPSO(Project SupportOffice)を設置
知識労働者の能力を最大限に活かすには,メンバーの専門知識と能力を結集して組織全体のミッションに貢献させることが必要である.そのため,Agile開発フレームワークScrumでは,開発チームは自己組織化されると考える,よって,プロジェクトのマネジメントは,管理監督では無く,プロジェクトの円滑な進行をサポートするサービスを提供するチームとした

PSO(Project SupportOffice)の機能
•	日本・ベトナム国家大学開発チームのコミュニケーションを円滑にするためのサポート
•	重要な意思決定要素は全員で協議
•	技術課題や仕様の課題の解決をサポート
•	日本側ステークホルダとのコミュニケーションサポート
•	日々のコミュニケーションとプロジェクト・モニタリング
•	プロセス保証
•	品質保証と品質確認
2)開発ゲーム 
開発チームは2つ設置し,同じソフトウェアの開発を競わせることで,自主的に考えプロジェクトに取り組むモチベーションを維持する.

\subsection{Agile開発フレームワークScrum}
開発プロセスの「アジャイル面」において中心となるコンセプト
•	短期開発
•	Time-box Management
SCRUMの短期タイムボックスマネジメントによって 設計~テストまでのスプリントを複数回実施する.1スプリントは,1W

要求変更,仕様変更
要求変更,仕様変更は不可避なものと考える.
しかし,変更は開発作業を止めると同時に,開発者のモチベーション(生産性)を下げることにつながるので, 生産性向上を狙いスプリント期間中は変更認めないもし変更があった場合は次回以後のスプリントでの対応を協議する.

自己組織化
設計以降の全権限を開発チームに委任し,自律性を重視した自由な開発をしてもらう

繰り返し型の開発プロセス(リスク管理、スコープ管理、教育効果)
SECIサイクルを何度も回す.
失敗が次のサイクルに活きる.PBL参加者の学習効果がUP

優先順位付け要求リスト(プロダクトバックログ)によってユーザにとって価値のある最小機能単位(価値の高い重要な機能)単位で開発する.これらをプロジェクトの途中段階から顧客へ届け,フィードバックを得ることで,開発上のリスクを低下させ、本当に必要な機能へ開発リソースを集中できるとともに、変更を次回の素プロンと取り込むかどうかを要求や仕様の優先度で判断できるため、要求や仕様の変更に柔軟に対応することが可能

Progress Management  バーンダウンチャート
 形式的な進捗報告会議を止め,「要求実現量の変化予測(バーンダウンチャート)」によって「進捗状況」を透明化し,共有する 

\subsection{開発スケジュールとプロジェクトの進め方の工夫(遅延対策)}
3回に分けた出荷計画(リリース計画),スプリント0とクリンナップスプリントを組み込んだ
スプリント期間は固定.遅それぞれの作業にバッファを与えず,全体のバッファの消化具合で進捗を管理する
、遅れを把握するした場合スコープ調整を行う

\subsection{スプリント0(リスク低減)}
アジャイル開発では,最初のイテレーション(スプリント)を「スプリント0」と呼ぶことがある.この期間では,重要な要件の洗い出しやプロトタイピング,ツールのセットアップ等を行い,スムーズに開発に入るための準備を行う.(プロジェクトによって準備事項は違う)
・重要な要求の確認
・アジャイル開発の準備
・マネジメント用資材の作成
・技術的課題の検証

\subsection{クリンナップスプリント(品質の保証}
・取りこぼしたバグの改修(掃除)
・リリーステスト (結合テスト,回帰テスト,等)
を行い品質向上を実施

\subsection{タイムバッファ}
Time Buffer Management
 タイムバッファの消費量をモニタリングすることで,プロジェクト全体の納期を達成するためのマネジメントを可能とする

\subsection{受け入れ評価試験Quality Gate(品質)}
 リリース前に「Quality Gate」を置くことで,品質管理活動を開発プロセスに組み込んだ
 Quality Gateでは,これまでの顧客の要求に基づく動作確認を行い,リリース可能な品質が確保されているかどうか確認する

\subsection{ACM: Asynchronous Communication Management :時差を乗り越えた非同期型コミュニケーション(コミュニケーション管理)}
日々のコミュニケーションとプロジェクト・モニタリングを円滑に行うため
ソーシャルメディア(Facebook)を利用したしくみの整備

\subsection{開発プロセスの実行支援(プロジェクトサポートサービス)}
毎週火曜・土曜日のAIIT側での進捗状況確認
最新のProduct Backlog,Sprint BacklogをFacebookにUPしてもらうことで,開発プロセスの進行状況を確認.問題があればフォローする.
課題管理・テクニカルサポートが必要になった場合は,下記の通り
•	仕様に関して
•	加藤先生からのサポート
•	開発環境に関して
•	1次対応:PBLメンバー
•	2次対応:加藤先生/土屋先生
•	その他の課題
•	PBLメンバー

\subsection{ソーシャル開発技術の活用による,ソースコード共有化}
Social Coding 環境を用い,ソースコードを共有化(メンバー全員が参照・コメント可能)にし,開発チーム全員誰もがソースコードの品質を確認し,問題があればコメントできるようにした.
さらに,AIITに動作確認環境を設置,品質や技術的な課題がないかチェックを行い,問題があれば,すぐさまFacebookでコミュニケーションをとり,問題の解決を行った

\subsection{Just Enough Documentation}
ドキュメントの目的は,他者に情報を伝えることと,現行仕様の文書化である.前者の目的に注力し最低限のドキュメントを作成する.特にアンドロイドOSの画面については,動作が確認できる画面をドキュメントの代わりとした.

Management As a Service
 
 アジャイル開発において,スプリント開発プロセスとスプリント開発チームにばかり焦点があたりがちだが,本当に重要なのはスプリント外における開発サポート活動にある.
 
「管理(プロジェクトマネジメント)」から「サポート」へ
 
 「後だし(フィードバック)」ではなく,「段取り(フィードフォワード)」へ
 
 プロジェクトマネージャの一番重要な役割は,外部のノイズから開発チームを守ることと,事前の計画に注力し,不確実性要素を早期発見し,段取りにより事前アクションを打つこと

\subsection{プロジェクトの実行}
 実行面における出来事,課題をどのように解決していったか,アジャイル開発フレームワークがプロジェクトへどのような効果を発揮したかを書く
\section{評価}
 プロジェクト完了結果と振り返りを書く
アンケート調査の結果をまとめる

\section{議論(KPT)}
 
\section{関連研究}

 
2.開発プロセスにAgile開発フレームワークScrumを選択

 2.2 Agile型の開発とは 
「アジャイル開発プロセス」とは,要件の変化に対応し,価値を追求する方法論です.
様々な手法が提案されていますが,いずれも共通的な価値観を共有しています
「アジャイル開発プロセス」は,ウォーターフォール型開発プロセスに比べて,
以下のメリットがあります.
ビジネスの視点
開発の視点

2.4 Agile開発フレームワークScrumとは?

1)Scrumの歴史

2)Scrumの理論
アジャイル・スクラム~ソフト開発プロジェクトは知識創造の場~

3)Scrumのプロセス

4)Scrumの主な適用場面


7. 国際分散アジャイル開発プロセスのオーバービュー


8.国際分散アジャイル開発プロセスのコンセプト
•	サービス指向プロジェクトマネジメント
•	Agile開発フレームワークScrum: 
•	ACM: Asynchronous Communication Management :時差を乗り越えた非同期型コミュニケーション
•	ソーシャル開発技術の活用
以上を融合したもの



8.(開発の結果(RSNPコンテストAPEN賞)


9.考察とまとめ~PBLを通して何を学んだか?

 






 3.4「アジャイルソフトウェア開発宣言」と価値
各アジャイル開発プロセスは,それぞれの価値に基づいたプラクティス集と言えますが,
様々なアジャイル開発プロセスの提唱者たちが,共通的な価値観に基づいていると気づき,共同で考えた価値と原則があります.(「アジャイルソフトウェア開発宣言」)
@@@アジャイル開発宣言@@@
プロセスやツールよりも個人と対話を,
包括的なドキュメントよりも動くソフトウェアを,
契約交渉よりも顧客との協調を,
計画に従うことよりも変化への対応を,
価値とする.すなわち,左記のことがらに価値があることを
認めながらも,私たちは右記のことがらにより価値をおく.

 3.5「アジャイルソフトウェア開発宣言」と原則
「アジャイルソフトウェア開発宣言」では12の原則も表明しています.
1.	顧客満足を最優先し,価値のあるソフトウェアを早く継続的に提供します.
2.	要求の変更はたとえ開発の後期であっても歓迎します.
変化を味方につけることによって,お客様の競争力を引き上げます.
3.	動くソフトウェアを,2-3週間から2-3ヶ月というできるだけ短い時間間隔でリリースします.
4.	ビジネス側の人と開発者は,プロジェクトを通して日々一緒に働かなければなりません.
5.	意欲に満ちた人々を集めてプロジェクトを構成します.
環境と支援を与え仕事が無事終わるまで彼らを信頼します.
6.	情報を伝えるもっとも効率的で効果的な方法はフェイス・トゥ・フェイスで話をすることです.
7.	動くソフトウェアこそが進捗の最も重要な尺度です.
8.	アジャイル・プロセスは持続可能な開発を促進します.
一定のペースを継続的に維持できるようにしなければなりません.
9.	技術的卓越性と優れた設計に対する不断の注意が機敏さを高めます.
10.	ンプルさ(ムダなく作れる量を最大限にすること)が本質です.
11.	最良のアーキテクチャ・要求・設計は,自己組織的なチームから生み出されます.
12.	チームがもっと効率を高めることができるかを定期的に振り返り,
それに基づいて自分たちのやり方を最適に調整します.
 3.6繰り返し型開発&小ロット開発
プラクティスにも共通的な概念があります.
特に主要な考え方が「繰り返し型開発」(反復型開発ともいう)と「小ロット開発」です

繰り返し型開発
2)	特徴
(ア)	短く,固定された期間(イテレーション)を繰り返しながら,
次第にソフトウェアを進化させて開発していく
(イ)	イテレーション毎に顧客からフィードバックを得ることで,
ビジネス価値を最大化させるように開発する要望を組み替え,計画を変更しつづける
3)	小ロット開発
4)	特徴
(ア)	小さく,顧客にとって意味のある単位にソフトウェアへの要望を分割して少しづつ開発していく
(イ)	繰り返し毎に要望の必要性を見直し,まだ開発していない要望と,現在での要望のビジネス価値を比較し,ビジネス価値が最大になるように計画を変更しつづける
5)	繰り返し型開発」と「小ロット開発」は,ソフトウェア開発の不確実性を解決するため,
少しずつ開発することでリスクを低減し,ビジネス価値を最大化させる方法です.

4 . Agile開発フレームワークScrum


開発チームの視点で
 考えられた方法論
・チームのマネジメントに
 関する考え方が中心
・近年のアジャイル開発で
 最も使われている方法論

アジャイル・スクラム
~ソフト開発プロジェクトは知識創造の場~
Initial Theory of Scrum
History of Scrum
Theory of Scrum
1.	Transparency
Significant aspects of the process be visible to those responsible for the outcome.
2.	Inspection
Scrum users must frequently  inspect Scrum artifacts and progress toward a goal
3.	Adaptation
The process must be adjusted, if the process deviate outside acceptable limits.




\section{おわりに}

%% 謝辞(ある場合のみ)
\begin{acknowledgements}
みんなありがとう!\cite{takeuchi1986new}
\end{acknowledgements}

%% 文献
\bibliographystyle{jplain}
\bibliography{reference}

%% 付録(ある場合のみ)
%\appendix 

%% 著者紹介(総合論文,学術・技術論文,解説論文のみ)
\begin{biography}
\profile{n}{酒瀬川 泰孝(Yasutaka Yasutaka))}{紹介文}
\profile{n}{川木 富美子(Tomiko Kawaki)}{紹介文}
\profile{n}{須澤 秀人(Hideto Suzawa)}{紹介文}
\profile{n}{木崎 悟(Hideto Suzawa)}{紹介文}
\profile{m}{土屋 陽介(Yosuke Tsuchiya)}{紹介文}
\profile{n}{中鉢 欣秀(Yoshihide Chubachi)}{情報処理学会.電子情報通信学会.日本マーケティング学会}
\end{biography}

\end{document}
                                                                                                                                                                                                                                                                                                                                                                                                                                                                                                                                                                           