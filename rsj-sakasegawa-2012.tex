%%% 論文(A)と一般記事用(B)のテンプレート
%%%   必要に応じてどちらかを使ってください.

%%% A. 論文
\documentclass[paper]{jrsj}
\usepackage{graphicx}

%% 頁(指定する必要なし)
%\setcounter{page}{1}
%% 通巻頁(指定する必要なし)
%\volpage{101}

%% 論文の種類を記入(学術・技術論文 / 学術論文 / 討論など)
\typeofpaper{学術・技術論文}
%% 発行年(コメントアウトしておく)
%\Year{2000}
%% 巻数(コメントアウトしておく)
%\Vol{18}
%% 号数(コメントアウトしておく)
%\No{1}
%% 邦文題名
\title{
国際教育のプロジェクトへのアジャイル開発方法論スクラムの適用とマネジメント(仮)
}
%% 邦文副題名(ある場合のみ)
\subtitle{}
%% 英文題名
\etitle{English Title}
%% 英文副題名(ある場合のみ)
\esubtitle{}
%% 著者のリスト
\authorlist{%
 \authorentry{酒瀬川 泰孝}{Yasutaka Yasutaka}{AIIT}
 \authorentry{川木 富美子}{Tomiko Kawaki}{AIIT}
 \authorentry{須澤 秀人}{Hideto Suzawa}{AIIT}
 \authorentry{木崎 悟}{Satoru Kizaki}{AIIT}
 \authorentry{土屋 陽介}{Yosuke Tsuchiya}{AIIT}
 \authorentry{中鉢 欣秀}{Yoshihide Chubachi}{AIIT}
}
%% 著者の所属
\affiliate[AIIT]{産業技術大学院大学}{Advanced Institute of Industrial Technology}

%% 原稿受付日(記述しない,空欄に)
\received{}

\begin{document}
% % 英文要旨
\begin{abstract}
Lorem ipsum dolor sit amet, consectetur adipiscing elit. Pellentesque sit amet nibh odio. Nam non bibendum velit. Vestibulum ante ipsum primis in faucibus orci luctus et ultrices posuere cubilia Curae; Ut sagittis elit ac sapien pulvinar pellentesque fermentum arcu aliquam. Integer sollicitudin lorem ac nulla mattis vehicula. Suspendisse sed mauris magna, a sagittis neque. Praesent egestas purus eu erat adipiscing in porta enim auctor. Vestibulum egestas neque in orci congue gravida. Pellentesque porttitor magna ac magna ornare pretium. Ut vestibulum sapien ut sem rutrum et suscipit velit consequat. Ut cursus libero et nisi faucibus pretium luctus erat sodales. Morbi lacus.
\end{abstract}
%% 英文キーワード
\begin{keywords}
Lorem ipsum dolor sit amet
\end{keywords}
\maketitle
\small
%% 本文
\section{はじめに}

産業のグローバル化が進み,ビジネスライフサイクルの超短期化,ビジネスピードがその企業のマーケットシェア決めている,ロボット開発においても,市場環境の変化の激化の下,製品開発のスピード競争が激化する,このため、製品開発のプロジェクトにはスピードと要求事項の変化への柔軟性という競合する要素を満たすことがより一層求められる.これが開発プロジェクトのマネジメントをより一層困難としている.このような国際開発プロジェクトを取り巻く環境の変化の中では,開発プロジェクトの成功に貢献する人材育成の成否が経営上の重要な課題である.
そこで,私たちは、グローバルな環境の下,プロジェクトを効果的にマネジメントし,要素技術を市場価値の高い製品へ統合できる開発プロジェクトのPMや技術者の育成が重要と考え,産業技術大学院大学(以下,AIIT)とベトナム国家大学ハノイ校(以下,VNU)と共同で毎年実施している,国際共同PBL(Project Based Learning)のテーマにロボットアプリケーションの開発を選定し,ロボットアプリケーション開発の国際開発プロジェクトを立ち上げ,これをPMや技術者の育成の場として教育に取り組んだ.その概念図を下記に示す.
加えて,製品開発のスピードと要求事項の変化への柔軟性という競合する要素を実現するための開発方法論としてシステム開発で注目を集めているアジャイル開発手法,中でもスクラム開発手法に着目しこれAIITとVNUとの国際ロボットアプリケーション開発の国際PBLに適用した.実施結果としては,納期通りにソフトウェアを満足のできる品質で開発できた.
本論文では, 第2章でPBLの目的と概要を,第3章でロボットを国際PBLに取り入れた目的や狙う教育効果ついて,4章では国際PBLで実施したロボットアプリケーション開発プロジェクトの最大の特徴である,アジャイル開発手法スクラムについて説明する.そして,5章では,VNUにおけるアジャイル開発手法スクラムのオンサイトトレーニング,6章では,国際ロボットアプリケーション開発プロジェクトの実施,7章では,国際ロボットアプリケーション開発プロジェクトの実行結果や得た知見を紹介する.最後に8章にて,本プロジェクトのプロジェクトマネジメントにアジャイル開発方法論スクラムを導入した結果や教育上の効果について考察加え報告する.

\section{国際ロボットアプリケーション開発プロジェクトの目的と概要}
\subsection{PBLの目的}
本プロジェクトの目的は,変化に柔軟に対応できる開発手法であるAgile開発方法論の代表的な手法のScrumに着目し,Scrum開発手法を適用したロボット制御プログラム開発プロジェクトをAIITとVNUで実施する.
そして,これを通して
アジャイル開発手法スクラムのロボットアプリケーション開発プロジェクトへ適用しその有用性を検討する
プロジェクト(演習)を通した学習で、実践的な知識の獲得や体験による対象の深い理解得ることを通して,国際開発プロジェクトのPMや技術者の育成を行う
上記の目的場として,2012年度のAIITとVNUとの国際共同PBLプログラムを活用した

\subsection{目指す人材像の設定}
この国際PBLでは,2国間でのAgile開発方法論の一つScrumを適用したロボットの制御ソフトウェア開発のプロジェクトを通して,以下の能力を獲得した人材の育成を目指した.
1)国際環境におけるロボットアプリケーション開発プロジェクトをマネジメントできる人材の育成,特にAgile開発の知識やプロジェクトマネジメント手法を習得し、開発プロジェクトにおける技術者やマネジメントの責任や役割,振る舞いを理解・習得する
2)異文化,異なる言語,多様な背景を持つ学生が同じPBLを実践し議論することで,多様な発想を持った人材の育成
3)上記の教育効果を実現するために必要な教材やメソッドの作成,これらを用いた教育の実施と検証を通してプロジェクトチームへの教育能力の獲得.
\subsection{育成方針}
AIITの院生は全員,IT業界での社会人経験を有しているため,このプロジェクトを通してグローバルプロジェクトのPMに必要な能力の獲得を目指す
VNUの学生学部2,3年生である.プログラミング言語や設計技法やツールについては習得済みであるが,プロジェクトでのチーム作業経験がない.そのため,プロジェクトを通し開発チームの技術者として業務を遂行するために必要のな能力の獲得とチーム開発作業の経験を獲得する.
\subsection{教育にロボットを使う狙い}
題材は,ロボットとインターネット,スマートフォンの融合をテーマに,RSNPプロトコルを利用しインターネット介して,Android端末を利用したロボット(LegoMindstorms)を操作するアプリケーションを選択した.教育にロボットを使う狙いは、以下の通りである
1)開発の課程・プロジェクト実践成果を可視化できる
2)実際に動くものを一緒に作ることで言葉の壁を越えた深い理解を得ることができる.
検証基準は、成果物(スプリントバッグログ/プロダクトバックログ/ソースコード/動作確認)、完了物を検証することとした.
% \subsection{グローバル環境のコミュニケーションツールの選択}
% グローバルな環境で開発プロジェクトを行う際の課題は,英語コミュニケーション能力と適切なコミュニケーションツール基盤の構築である.
% AIITの院生は仕事を通して全員十分な英語力を持っていたため問題はなく,VNUの学生も授業が英語で行われているため極めて高い英語力をもっていた.この課題はクリアできた.
% 残るは,コミュニケーション基盤の構築である.時間と距離を超えた主なコミュニケーション手段にeメールがあるが
% これは,宛先以外に進捗などの重要な情報が共有できないため,プロジェクトの見える蚊には適さしにくいと考えた.また,その他のメールに埋もれてしまい,発見や返信が遅れるなどの点が懸念された,そこで我々は,ソーシャルメディアであるFacebookに注目し,これをプロジェクトのすべての期間を通してコミュニケーションに活用した,特にファイルが添付できる点や,発言に全員がコメントし議論を活性化できる点に注目した. 
% \item 

\section{アジャイル開発手法}
アジャイル開発手法は,ソフトウェア開発の分野で注目されている,開発手法であるソ.フトウェアが「柔軟で変更に強く」なるように開発プロセスがデザインされた開発方法論の総称である.   
我々は,スピードと要求事項の変化への柔軟性という競合する要素を満たすことが求められる今日のロボット開発にも有効であると考えた. 
アジャイル開発手法は下記の特徴がある. これらは,ソフトウェア開発の不確実性を解決するため,少しずつ繰り返し開発することでソフトウェア開発の不確実性を解決しながらリスクを低減し,生み出すソフトの価値を最大化させる考え方である
1)  繰り返し型開発
短く,固定された開発期間を定める.これをイテレーションと呼ぶ.イテレーションを繰り返しながら,漸進的にソフトウェアを開発をすすめる.イテレーション毎に顧客からフィードバックを得ることで,ソフトウェアのビジネス上の価値を最大化させる.顧客からの要求やフィードバックに基づき,開発の要件を組み替え,計画を変更しつづける.
2)	小ロット開発
小さく,顧客にとって意味のある単位にソフトウェアへの要望を分割して少しずつ開発する.イテレーション毎に要望の必要性を見直し,まだ開発していない要望と,現在での要望のフトウェアのビジネス上の価値を比較しソフトウェアのビジネス上の価値が最大になるように計画を変更しつづける

そして,我々は数あるアジャイル開発手法の中から,スクラムを採用した.これは,スクラムが日本人研究者による日本の製造業の研究成果が起源になっており,マネジメントの枠組みが他のアジャイル開発手法に比べ確立され, また,ソフトウェア開発の分野でも最も多くの開発プロジェクトで多くの事例がある.

\section{アジャイル開発手法スクラム}
\subsection{スクラム}
スクラムの起源は,竹内弘高・野中郁次郎の論文による.これは、1980年代の日本の製造業の新製品開発のプロセスを研究し,「製品開発におけるリレー競走のアプローチは、最大のスピートと柔軟性というゴールとは競合するだろう。代わりに、 全体的もしくはラグビーのようなアプローチ(チームは離れたゴールまで塊として進もうとし、ボールを後ろへ前へパスする)は、今日の競争力のある要求の実現にとって、より役立つだろう」と述べ,逐次型の開発プロセスより繰り返し型のプロセスが有効とした.
これは、現在のソフトウェア開発におけるウォーターフォール型開発とアジャイル開発の比較にも通じる。
そして、竹内・野中はこのようなチームを「スクラム」と名付けた。、チームの6つの特徴を挙げた。これらは、アジャイル開発手法のスクラムにも引き継がれている.
1)不安定な状態を保つ
メンバには高い自由裁量と同時に、極端に困難なゴールを与える
2)プロジェクトチームは自ら組織化する
設立したばかりの企業のように、「情報ゼロ」の状態から始めると、メンバは自律、自己超越、相互交流を自ずと始める
3)開発フェーズを重複させる
開発フェーズを重複させることで、「分業の共有」という状態を作り出し、メンバはプロジェクト全体に責任感をもつようになる
4)マルチ学習
メンバの学習は、チーム中に多層と機能の2つのレベルで学び」が起こることをマルチ学習と呼んでいる。「多層」(個人、グループ、組織、企業といった複数のレベルで学習がおこること)と「多機能学習」(別々のスキルを持った人が集まることで、専門外の知識についての学習が起こること)の2つである。
5)巧みにマネージするマルチ学習
放任せず、自己管理、メンバ間管理、と愛情による管理を強調する
6)学びを組織で共有する
過去の成功・失敗からの学びの習得・忘却を組織内で浸透させる

\section{国際ロボットアプリケーション開発プロジェクト}
\section{プロジェクトの実行}
\section{プロジェクト結果の考察と振り返り}
\section{おわりに}

%% 謝辞(ある場合のみ)
\begin{acknowledgements}
産業技術大学院大学 大学情報アーキテクチャ専攻で主担当してご指導いただいた中鉢先生に感謝する,副担として議論くださった,酒森教授,加藤教授,土屋助教,長尾助教の各先生に感謝する.
加えて,ともにプロジェクトに関わったベトナム国家大学ハノイ校の皆様に謝意を表明する.
Truong Anh Hoang 教授,Thi-Minh-Chau TRAN教授, Manh-Cuong NGUYEN氏, Duc-Kien DO氏, Dinh-Nien NGUYEN氏, hac-Phong DO氏,Hung-Quan TRAN氏, Xuan-Thuy氏,DONG,Dinh-Vuong PHAM氏, Manh-Toan NGUYEN,Xuan-Hoa NGO氏, Xuan-Thanh NGUYEN氏
最後に,ともに議論しともに学んだ須澤氏と川木氏に感謝の意を表します.
\cite{takeuchi1986new}
\end{acknowledgements}

%% 文献
\bibliographystyle{jplain}
\bibliography{reference}

%% 付録(ある場合のみ)
%\appendix 

% % 著者紹介(総合論文,学術・技術論文,解説論文のみ)
\begin{biography}
\profile{n}{酒瀬川 泰孝(Yasutaka Yasutaka))}{産業技術大学院大学::株式会社NTTデータ.。PMI(米国プロジェクトマネジメント協会)会員.PMI日本支部会員,同PMBOKセミナーグループメンバ.プロジェクトマネジメント学会正会員}
\profile{n}{川木 富美子(Tomiko Kawaki)}{紹介文}
\profile{n}{須澤 秀人(Hideto Suzawa)}{紹介文}
\profile{n}{木崎 悟(Hideto Suzawa)}{紹介文}
\profile{m}{土屋 陽介(Yosuke Tsuchiya)}{紹介文}
\profile{n}{中鉢 欣秀(Yoshihide Chubachi)}{情報処理学会.電子情報通信学会.日本マーケティング学会}
\end{biography}

\end{document}
                                                                                                                                                                                                                                                                                                                                                                                                                                                                                                                                                                           