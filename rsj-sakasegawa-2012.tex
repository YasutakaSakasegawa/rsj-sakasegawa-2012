%%% 論文(A)と一般記事用(B)のテンプレート
%%%   必要に応じてどちらかを使ってください.

%%% A. 論文
\documentclass[paper]{jrsj}
\usepackage{graphicx}

%% 頁(指定する必要なし)
%\setcounter{page}{1}
%% 通巻頁(指定する必要なし)
%\volpage{101}

%% 論文の種類を記入(学術・技術論文 / 学術論文 / 討論など)
\typeofpaper{学術・技術論文}
%% 発行年(コメントアウトしておく)
%\Year{2000}
%% 巻数(コメントアウトしておく)
%\Vol{18}
%% 号数(コメントアウトしておく)
%\No{1}
%% 邦文題名
\title{
国際教育のプロジェクトへのアジャイル開発方法論スクラムの適用とマネジメント(仮)
}
%% 邦文副題名(ある場合のみ)
\subtitle{}
%% 英文題名
\etitle{English Title}
%% 英文副題名(ある場合のみ)
\esubtitle{}
%% 著者のリスト
\authorlist{%
 \authorentry{酒瀬川 泰孝}{Yasutaka Yasutaka}{AIIT}
 \authorentry{川木 富美子}{Tomiko Kawaki}{AIIT}
 \authorentry{須澤 秀人}{Hideto Suzawa}{AIIT}
 \authorentry{木崎 悟}{Satoru Kizaki}{AIIT}
 \authorentry{土屋 陽介}{Yosuke Tsuchiya}{AIIT}
 \authorentry{中鉢 欣秀}{Yoshihide Chubachi}{AIIT}
}
%% 著者の所属
\affiliate[AIIT]{産業技術大学院大学}{Advanced Institute of Industrial Technology}

%% 原稿受付日(記述しない,空欄に)
\received{}

\begin{document}
% % 英文要旨
\begin{abstract}
Lorem ipsum dolor sit amet, consectetur adipiscing elit. Pellentesque sit amet nibh odio. Nam non bibendum velit. Vestibulum ante ipsum primis in faucibus orci luctus et ultrices posuere cubilia Curae; Ut sagittis elit ac sapien pulvinar pellentesque fermentum arcu aliquam. Integer sollicitudin lorem ac nulla mattis vehicula. Suspendisse sed mauris magna, a sagittis neque. Praesent egestas purus eu erat adipiscing in porta enim auctor. Vestibulum egestas neque in orci congue gravida. Pellentesque porttitor magna ac magna ornare pretium. Ut vestibulum sapien ut sem rutrum et suscipit velit consequat. Ut cursus libero et nisi faucibus pretium luctus erat sodales. Morbi lacus.
\end{abstract}
%% 英文キーワード
\begin{keywords}
Lorem ipsum dolor sit amet
\end{keywords}
\maketitle
\small
%% 本文
\section{はじめに}

産業のグローバル化が進み,ビジネスライフサイクルの超短期化,ビジネスピードがその企業のマーケットシェア決めている,ロボット開発においても,市場環境の変化の激化の下,製品開発のスピード競争が激化する,このため、製品開発のプロジェクトにはスピードと要求事項の変化への柔軟性という競合する要素を満たすことがより一層求められる.これが開発プロジェクトのマネジメントをより一層困難としている.このような国際開発プロジェクトを取り巻く環境の変化の中では,開発プロジェクトの成功に貢献する人材育成の成否が経営上の重要な課題である.
そこで,私たちは、グローバルな環境の下,プロジェクトを効果的にマネジメントし,要素技術を市場価値の高い製品へ統合できる開発プロジェクトのPMや技術者の育成が重要と考え,産業技術大学院大学(以下,AIIT)とベトナム国家大学ハノイ校(以下,VNU)と共同で毎年実施している,国際共同PBL(Project Based Learning)のテーマにロボットアプリケーションの開発を選定し,ロボットアプリケーション開発の国際開発プロジェクトを立ち上げ,これをPMや技術者の育成の場として教育に取り組んだ.その概念図を下記に示す.
加えて,製品開発のスピードと要求事項の変化への柔軟性という競合する要素を実現するための開発方法論としてシステム開発で注目を集めているアジャイル開発手法,中でもスクラム開発手法に着目しこれAIITとVNUとの国際ロボットアプリケーション開発の国際PBLに適用した.実施結果としては,納期通りにソフトウェアを満足のできる品質で開発できた.
本論文では, 第2章でPBLの目的と概要を,第3章でロボットを国際PBLに取り入れた目的や狙う教育効果ついて,4章では国際PBLで実施したロボットアプリケーション開発プロジェクトの最大の特徴である,アジャイル開発手法スクラムについて説明する.そして,5章では,VNUにおけるアジャイル開発手法スクラムのオンサイトトレーニング,6章では,国際ロボットアプリケーション開発プロジェクトの実施,7章では,国際ロボットアプリケーション開発プロジェクトの実行結果や得た知見を紹介する.最後に8章にて,本プロジェクトのプロジェクトマネジメントにアジャイル開発方法論スクラムを導入した結果や教育上の効果について考察加え報告する.

\section{国際ロボットアプリケーション開発プロジェクトの目的と概要}
\section{アジャイル開発手法}
\section{アジャイル開発手法スクラム}
\section{国際ロボットアプリケーション開発プロジェクト}
\section{プロジェクトの実行}
\section{プロジェクト結果の考察と振り返り}
\section{おわりに}

%% 謝辞(ある場合のみ)
\begin{acknowledgements}
みんなありがとう!\cite{takeuchi1986new}
\end{acknowledgements}

%% 文献
\bibliographystyle{jplain}
\bibliography{reference}

%% 付録(ある場合のみ)
%\appendix 

%% 著者紹介(総合論文,学術・技術論文,解説論文のみ)
\begin{biography}
\profile{n}{酒瀬川 泰孝(Yasutaka Yasutaka))}{産業技術大学院大学::株式会社NTTデータ.。PMI(米国プロジェクトマネジメント協会)会員.PMI日本支部会員,同PMBOKセミナーグループメンバ.プロジェクトマネジメント学会正会員}
\profile{n}{川木 富美子(Tomiko Kawaki)}{紹介文}
\profile{n}{須澤 秀人(Hideto Suzawa)}{紹介文}
\profile{n}{木崎 悟(Hideto Suzawa)}{紹介文}
\profile{m}{土屋 陽介(Yosuke Tsuchiya)}{紹介文}
\profile{n}{中鉢 欣秀(Yoshihide Chubachi)}{情報処理学会.電子情報通信学会.日本マーケティング学会}
\end{biography}

\end{document}
                                                                                                                                                                                                                                                                                                                                                                                                                                                                                                                                                                           