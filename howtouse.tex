%%
\documentclass[paper]{jrsj}    %% 論文タイプ
%\documentclass[article]{jrsj} %% 一般記事タイプ
\usepackage{graphicx}
\usepackage{latexsym}
%\usepackage{amsmath}

%\setcounter{page}{1}
%\volpage{101}

\typeofpaper{解説論文}
\typeofarticle{解説}
%\Year{2000}
%\Vol{18}
%\No{1}
\title[日本ロボット学会誌\LaTeXe{}クラスファイルの使い方]
      {日本ロボット学会誌\LaTeXe{}クラスファイルの使い方\\
       (ver.1.1 2000年5月22日版)}
\etitle{How to Use ``\JRSJcls'' Class File\\
        for the Journal of the Robotics Society of Japan (ver.1.1 2000.5.22)}
\authorlist{%
 \authorentry{会誌編集委員会}{Journal Editorial Committee}{JRSJ}
 \authorentry{論文査読小委員会}{Refereeing Sub Committee}{JRSJ}
}
\makeatletter\if@paper\makeatother
 \affiliate[JRSJ]{日本ロボット学会}{The Robotics Society of Japan}
\else
 \affiliate[JRSJ]{日本ロボット学会}{The Robotics Society of Japan}
  {文京区本郷 2--19--7}{2--19--7 Hongo, Bunkyo-ku, Tokyo}
\fi
\received{2000年5月22日}

\makeatletter\if@paper\makeatother\else
 \keyword{Class file, ASCII p\LaTeXe{}, Typesetting}% . NTT版\JTeX{}.
\fi

%% <local definitions>
\def\JRSJcls{{\ttfamily jrsj.cls}}
\def\PS{{\scshape Post\-Script}}
\def\AmSLaTeX{%
 $\mathcal{A}$\lower.4ex\hbox{$\!\mathcal{M}\!$}$\mathcal{S}$-\LaTeX}
\makeatletter\if@paper\makeatother
 \def\verbatimsize{\small}%%??
\fi
%% </local definitions>

\begin{document}
\makeatletter\if@paper\makeatother
\begin{abstract}
 The Robotics Society of Japan has designed \LaTeXe{} class file
 named \JRSJcls{} for the Journal of Robotics Society of Japan. 
 It can be used both for papers and articles 
 (Kouza, Kaisetsu, Tembou and so on).
 This document describes how to use the class file, 
 and also makes some remarks about typesetting a document by using \LaTeXe{}. 
 This class file is fundamentally designed based on ASCII p\LaTeXe{}.
 This document itself is an example of the usage of this class file.
\end{abstract}
\begin{keywords}
 Class file, ASCII p\LaTeXe{}, Typesetting
\end{keywords}
\fi
\maketitle

\section{はじめに}\label{sec:prologue}

論文および一般記事の執筆上にかかわる注意事項は,
「論文原稿作成要領」を参照してください.
ここでは,クラスファイルの使用にかかわる点のみを説明します.

\JRSJcls{} クラスファイルは,オプションを指定することにより,
論文(総合論文,学術・技術論文,解説論文,
研究速報,討論)および一般記事(随想,展望,解説,講座,講演など)の
両方のフォーマットで利用できます.

「論文」は,本文の活字の大きさを,写植の単位の12級($3 \times 3$\,mm の
大きさの文字,9\,pt 相当)に設定しています.また,
「一般記事」は,活字の大きさを,13級($3.25 \times 3.25$\,mm の
大きさの文字,10\,pt 相当)に設定しています.
本文のサイズ指定は,「論文」は \verb/\small/,
「一般記事」は \verb/\normalsize/ となっています.

本誌の組版ルールに従って,各種パラメータと出力形式を変更しています.
したがって,レイアウトに関係するパラメータは
変更しないでください.

\section{クラスファイルの説明}

ドキュメントクラスのオプションに,
「論文」の場合は {\ttfamily paper} を,
「一般記事」の場合は {\ttfamily article} を,それぞれ指定します.
何も指定しない場合は,
{\ttfamily paper} が指定されたものとみなします.

\subsection{テンプレートと記述方法}

「論文」の場合と「一般記事」の場合の記述方法をそれぞれ,
\ref{sec:paper} 項と \ref{sec:article} 項で説明します.
また,英文原稿の場合の記述方法は \ref{sec:english} 項で説明します.

「論文」形式と「一般記事」形式の違いは,
それぞれの種類を記述する部分,
著者の所属の記述,キーワードの記述の部分です.

執筆に際しては,本クラスファイルとともに配布されている
テンプレート({\ttfamily template.tex})を利用できます.

\subsubsection{「論文」形式}\label{sec:paper}

\begin{verbatim}
\documentclass[paper]{jrsj}
\usepackage{graphicx}
\typeofpaper{学術・技術論文}
%\Year{2000}
%\Vol{18}
%\No{1}
\title{邦文題名}
\subtitle{邦文副題名}
\etitle{英文題名}
\esubtitle{英文副題名}
\authorlist{%
 \authorentry{ロボ 太郎}{Robo Taro}{Tokyo}
 \authorentry{学会 花子}{Gakkai Hanako}{Osaka}
}
\affiliate[Tokyo]{日本ロボット学会}
  {Robotics Society of Japan}
\affiliate[Osaka]{大阪ロボット学会}
  {Robotics Society of Osaka}
\received{2000年5月22日} 
\begin{document}
\begin{abstract}
 英文要旨
\end{abstract}
\begin{keywords}
 英文キーワード
\end{keywords}
\maketitle
 本文
\begin{acknowledgements}
 謝辞
\end{acknowledgements}
\begin{thebibliography}{99}
\bibitem{1}
 ...
\end{thebibliography}
\begin{biography}
\profile{m}{名前(Name)}{紹介文}
\end{biography}
\end{verbatim}

\begin{itemize}
\item
\verb/\typeofpaper{}/ には論文の種類,
「総合論文」,「学術・技術論文」,「解説論文」,「研究速報」,
「討論」のどれかを記述します.

\item
\verb/\Year{}/,\verb/\Vol{}/,\verb/\No{}/ は,
発行年,巻数,号数を指定します.これらは柱に出力されます.
それぞれアラビア数字を記述します.

論文執筆時に掲載される号が未定の場合は,
コメントアウトしておいてください.

\item
\verb/\title{}/ には邦文題名を記述します.
任意の場所で改行する場合は,\verb/\\/ で改行します.

引き数は,柱にも出力されます.
タイトルが長すぎて柱の文字がはみ出す場合(ワーニングが出力されます)などは,
\begin{verbatim}
\title[短い邦文題名]{邦文題名}
\end{verbatim}
という形で,柱用に短くした邦文題名を記述できます.

必要に応じて,邦文副題名を \verb/\subtitle{}/ に記述できます.
これは必須ではありません.

\item
\verb/\etitle/ には,英文題名を記述します.
英文題名は柱には出力されないので \verb/\etitle[短い英文題名]{英文題名}/ と
いう使い方はできません.

必要に応じて,英文副題名を \verb/\esubtitle{}/ に記述できます.
これは必須ではありません.

\item
著者名は次のようなフォーマットで記述します.
和文・英文の著者名,和文・英文の所属などの
出力体裁を自動的に整えます.
\begin{verbatim}
\authorlist{%
 \authorentry{和文著者名}{英文著者名}{所属ラベル}
}
\end{verbatim}
という形です.例えば,以下のように記述します.
\begin{verbatim*}
\authorlist{%
\authorentry{ロボ 太郎}{Robo Taro}{Tokyo}
\authorentry{学会 花子}{Gakkai Hanako}{Osaka}
}
\end{verbatim*}
記述に際していくつかの注意があります.
 \begin{itemize}
 \item
 第1引き数の和文著者名の
 {\bfseries 姓と名の間には必ず半角のスペースを挿入}します
 (スペースを挿入し忘れた場合にはワーニングが出力されます).

 \verb/\authorentry/ の和文著者名は柱にも出力されます.

 \item
 第2引き数の英文著者名は,頭文字のみ大文字で記述します.

 \item
 第3引き数には,所属ラベルを記述します.
 このラベルは,後述の \verb/\affiliate/ の第1引き数に対応します.
 ラベルは大学名,企業名,地名などを表す短く簡潔なものにします.

 著者に所属がない場合は,{\ttfamily none} と指定します.
 また,2か所の所属がある場合には,ラベルをカンマ ``,'' で区切ります.

 ラベルの前後やカンマの後ろにスペースを挟まないでください.
 \verb/{Tokyo}/ と \verb*/{Tokyo }/ は所属を違うものと判断します.

 \item
 \verb/\authorlist/ マクロは,著者名や所属の出力体裁を
 自動的に整えますが,著者が多数の場合などに,任意の場所で
 改行を行いたい場合は,
 \verb/\breakauthorline/ コマンドが使用できます.
 これは{\bfseries 英文著者名だけ}に有効です.
 \verb/\breakauthorline{3}/ とすれば3人目の英文著者名の後ろで改行します.

 また,\verb/\breakauthorline{2,4,6}/ とすれば,
 2番目,4番目,6番目の英文著者名の後ろで改行します.
 \end{itemize}

\item
所属は,次のようなフォーマットで記述します.
\begin{verbatim}
\affiliate[所属ラベル]{邦文所属}{英文所属}
\end{verbatim}
これらの情報は,脚注部分に出力されます.

記述に際していくつかの注意があります.
 \begin{itemize}
 \item
 第1引き数は \verb/\authorentry/ で指定したラベルを記述します.
 第2引き数には邦文の所属を,第3引き数には英文の所属を記述します.
 \item
 ラベルの前後に余分なスペースを挿入しないでください.
 \item
 \verb/\authorentry/ で記述したラベルの出現順に記述してください.
 \end{itemize}
\item
\verb/affiliate/ のラベルが,
\verb/\authorentry/ で記述したラベルと対応しないときは,
ワーニングメッセージが端末に出力されます.
\item
\verb/\received{}/ には,原稿受付日を記述します.
原稿受付日は学会が記入するため,
引き数は空にしておいてください.
\item
英文要旨は abstract 環境に,
英文キーワードは keywords 環境にそれぞれ記述します.
\item
\verb/\maketitle/ は keywords 環境の直後に記述します.
\item
謝辞がある場合には,acknowledgements 環境に記述します.
\item
参考文献は,thebibliography 環境に記述します.
\item
付録が必要な場合は,\verb/\appendix/ コマンドを記述した後に
本文を記述してください(\pageref{sec:app} ページ参照).
数式番号は ``(A.1)'' のようになります.
\item
著者紹介は,「総合論文」,「学術・技術論文」,「解説論文」に必要です.
以下のようなフォーマットで記述します.
\begin{verbatim}
\begin{biography}
\profile{m}{名前(Name)}{紹介文}
\profile{n}{名前(Name)}{紹介文}
\end{biography}
\end{verbatim}
\verb/\profile/ の第1引き数に,ロボット学会会員資格の別を
記述します.{\ttfamily m} と記述すると ``(日本ロボット学会正会員)'' が,
{\ttfamily s} と記述すると ``(日本ロボット学会学生会員)'' が,
それぞれ紹介文の最後に追加されます.
{\ttfamily n} と記述した場合は何も追加されません.
第2引き数には名前(全角括弧で英文名を続ける)を,
第3引き数には紹介文をそれぞれ記述します.
\end{itemize}

\subsubsection{「一般記事」形式}\label{sec:article}

\begin{verbatim}
\documentclass[article]{jrsj}
\usepackage{graphicx}
\typeofarticle{解説}% 
%\Year{2000}
%\Vol{18}
%\No{1}
\title{}
\subtitle{}
\etitle{}
\esubtitle{}
\authorlist{%
 \authorentry{姓 名}{First Family}{label}
 \authorentry{}{}{}
}
\affiliate[label]{邦文所属}{英文所属}
 {邦文住所}{英文住所}
\received{}
\keyword{}
\begin{document}
\maketitle
%% 本文
\begin{thebibliography}{99}
\bibitem{}
\end{thebibliography}
\begin{biography}
\profile{m}{名前(Name)}{紹介文}
\profile{n}{名前(Name)}{紹介文}
\end{biography}
\end{verbatim}

「論文」形式と異なる部分について説明します.
所属の記述については,「論文」形式と同じにすると
エラーになりますから,注意してください.
\begin{itemize}
\item
\verb/\typeofarticle{}/ には,一般記事の種類,
「随想」,「展望」,「解説」,「講座」,「講演」などを記述します.
\item
所属は,次のようなフォーマットで記述します.
邦文住所と英文住所を記述するため,引き数が5つに増えます.
\begin{verbatim}
\affiliate[所属ラベル]{邦文所属}{英文所属}
 {邦文住所}{英文住所}
\end{verbatim}
\item
キーワードは \verb/\keyword{}/ に記述します.
「論文」形式で使った keywords 環境と混同しないでください.
\end{itemize}

\subsubsection{英文の場合}\label{sec:english}

ドキュメントクラスのオプションとして {\ttfamily english} を指定します.
\begin{verbatim}
\documentclass[article,english]{jrsj}
\end{verbatim}

\begin{itemize}
\item
英文題名と副題名は,それぞれ
\verb/\title{}/,\verb/\subtitle{}/ に記述します.
\verb/\etitle{}/,\verb/\esubtitle{}/ は記述しても無効になります.

\item
\verb/\authorentry/ の第1引き数(和文著者名)に記述しても無効になります.
空でかまいません.

\item
\verb/\affiliate/ の邦文所属と邦文住所は記述しても無効になります.
空でかまいません.
\end{itemize}

\subsection{クラスファイルの特徴と注意事項}

\subsubsection{見出しの字どり}

章見出し(\verb/\section/)は,見出しが6字以下の場合,7字どりになり
(\ref{sec:prologue} 章などを参照),センタリングされます.
任意の場所で改行したい場合は,``\verb/\\/'' で
折り返してください.

\subsubsection{別行立て数式}

\begin{itemize}
\item
別行立て数式は,通常のセンタリングで出力されます.

\item
数式番号は右端から1字入ったところに出力されます.
本誌では,数式番号は全角括弧が使用されていますので,
本文中で数式を参照する場合は,半角括弧ではなく,
全角括弧を使用してください(式\verb/(\ref{eq:1})/参照).

\item
本誌の場合は2段組みで,
1段の左右幅がせまいため数式と数式番号が重なったり,
数式がはみ出したりすることが頻繁に生じると思われます.
\verb/Overfull \hbox/ のメッセージに気をつけてください.

\item
数式の記述に関しては,\ref{sec:eq1} 節および \ref{sec:eq2} 節でも
説明しています.
\end{itemize}

\subsubsection{定理,定義などの環境}

定理,定義,命題などの定理型環境は \verb/\newtheorem/ が
利用できます\cite{latex}.
標準のクラスファイルでは環境中の欧文がイタリックになりますが,
本クラスファイルでは,イタリックにならないように変更しています.

たとえば,
\begin{verbatim}
\newtheorem{theorem}{定理}
\begin{theorem}
$n>2$ に対しては,
方程式 $x^n + y^n = z^n$ の
自然数解は存在しない
(Fermat's last theorem).
\end{theorem}
\end{verbatim}
と記述すれば,
\newtheorem{theorem}{定理}
\begin{theorem}
$n>2$ に対しては,
方程式 $x^n + y^n = z^n$ の
自然数解は存在しない
(Fermat's last theorem).
\end{theorem}
と出力されます.

「定理」に番号をつけたくない場合は,例えば,
上のように theorem が定義されているとすると,
\begin{verbatim}
\let\thetheorem\relax
\end{verbatim}
と記述すれば番号がつきません.

\subsubsection{図表とキャプション}

\paragraph{図表の記述}

\begin{itemize}
 \item
 図は基本的に\PS{} 形式を利用してください.

 例えば,パッケージとして
\begin{verbatim}
\usepackage[dvips]{graphicx}
\end{verbatim}
 を指定し,
\begin{verbatim}
 \begin{figure}[tb]
 \begin{center}
  \includegraphics{file.eps}
 \end{center}
 \caption{「論文」形式の場合は,英文キャプション}
 \label{fig:1}
 \end{figure}
\end{verbatim}
 のように記述します.
 \verb/\caption/ は図の要素の下に記述します.
 詳しくは,文献\cite{Nakano,Okumura3,Eguchi,Takayama}などを
 参照してください.

 \item
 表は \verb/\footnotesize/(8\,pt,11級)で組まれるように設定しています.
 以下のように記述します.
\begin{verbatim}
\begin{table}[tb]
\caption{「論文」形式の場合は,英文キャプション}
\label{table:1}
\begin{center}
 \begin{tabular}[t]{|c|c|c|}
 \hline
  A & B & C \\
 \hline
 \end{tabular}
\end{center}
\end{table}
\end{verbatim}
 \verb/\caption/ は tabular 環境の前に記述します.

 \item
 \verb/\label/ を記述する場合は,必ず
 \verb/\caption/ の直後に置いてください.前におくと \verb/\ref/ で
 正しい番号が参照できません.

 \item
 図表の出力位置を指定するオプションとして,{\ttfamily [h]} は使わず,
 {\ttfamily [t]},{\ttfamily [b]},{\ttfamily [tbp]} などを指定して,
 ページの天か地に出力させるようにしてください.
\end{itemize}

\paragraph{キャプションについて}

\begin{itemize}
 \item
 キャプションは,中央揃えで出力されます.
 \item
 2段抜きの図表のキャプションの場合,キャプションの幅をテキストの
 幅の 2/3 の長さで折り返すように設定しています.
 任意の場所で改行したい場合は,
 ``\verb/\\/'' で折り返すことができます.標準の \LaTeXe{} では
 こういう使い方はできませんので注意してください.
 \item
 任意の長さで折り返したい場合は,\verb/\caption/ の前で
\begin{verbatim}
 \capwidth=60mm
\end{verbatim}
 と記述すれば,60\,mm の長さで折り返します.
\end{itemize}

\subsubsection{脚注と脚注マーク}\label{sec:footnote}

脚注%
\footnote{「論文原稿作成要領」によれば,
「脚注はないことが望ましい」とされています.}%
が必要な場合は,\verb/\footnote/ を利用してください.
脚注マークは ``$^\dagger$'',``$^{\dagger\dagger}$'' という形で出力されます.

\subsubsection{verbatim 環境}

verbatim 環境のレフトマージン,行間,サイズを
変更することができます\cite{bibunsho2}.デフォルトは
\begin{verbatim}
\verbatimleftmargin=0pt
 % レフトマージンは 0pt 
\def\verbatimsize{\small}
 % フォントサイズ(「論文」形式の場合)
\def\verbatimsize{\normalsize}
 % フォントサイズ(「一般記事」形式の場合)
\verbatimbaselineskip=\baselineskip
 % 本文と同じ行間
\end{verbatim}
ですが,それぞれパラメータやサイズ指定を変更することができます.
\begin{verbatim}
\verbatimleftmargin=2zw
% --> レフトマージンを2字下げに変更
\def\verbatimsize{\footnotesize}
% --> サイズを \footnotesize に変更
\verbatimbaselineskip=3mm
% --> 行間を 3mm に変更
\end{verbatim}

\subsubsection{文献の引用と thebibliography 環境}

文献引用の \verb/\cite{}/ は,
{\ttfamily cite.sty} および {\ttfamily citesort.sty} に手を加えたものを
使用しています.例えば,\verb/\cite{Bech,Gr,/\allowbreak
 \verb/itou,ohno,nodera1,Abrahams,tex}/ とすれば,
次のように,番号が続く場合は省略し,番号順に並べ変えます
\cite{Bech,Gr,itou,ohno,nodera1,Abrahams,tex}.

参考文献の著者名,文献名,ジャーナル(出版社),発行年など,
イニシャル,略語のスタイル,順番などは本誌の規則に従ってください.

参考文献のリストは thebibliography 環境を使用し,
次のような記述例に従ってください.

\begin{verbatim}
\begin{thebibliography}{99}
\bibitem{jbook}
著者:書名.引用ページ,出版社,発行年.
%(和文著書の例)
\bibitem{ebook}
Author(s): Book Title. pp.XX--YY, publisher, year. 
%(英文著書の例)
\bibitem{jpaper}
著者: ``題名'',掲載誌名,巻,号,ページ,発行年.
%(和文論文の例)
\bibitem{epaper}
Author(s): ``Title,'' 
Name of Journal, vol.W, no.X, pp.YYY--ZZZ, year. 
%(英文論文の例)
% 記 載 例
\bibitem{Takahasi93}
高橋,吉田,坪内,木下:
``投稿原稿作成の手引'',
日本ロボット学会誌,vol.11, no.7, pp.88--99, 1993. 
\bibitem{Yoshida93}
K.~Yoshida, T.~Tsubouchi and G.~Kinoshita: 
``Instruction of making your manuscript,'' 
J. of the Robotics Society of Japan, 
vol.11, no.7, pp.88--99, 1993.  
\bibitem{Yoshida93b}
吉田,坪内: 
``論文作成のしおり'',
日本ロボット学会第11回学術講演会予稿集,
pp.1--5, 1993. 
\bibitem{Takahashi94}
K.~Takahashi and G.~Kinoshita: 
Guide Lines for Writing Your Paper. pp.6--10, 
The Robotics Society Press, 1994.
\bibitem{Asama94} 
H.~Asama: 
``Fundamental writing,'' 
Proc.\ of Int.\ Conf.\ Technical Writing, 
Tokyo, Japan, Apr.\ 1994, pp.2001--2006.
\end{verbatim}

\subsubsection{\JRSJcls{} で定義しているマクロ}

\begin{enumerate}
\item
\verb/\onelineskip/,\verb/\halflineskip/ という行間スペースを
定義しています.
その名のとおり,1行空け,半行空けに使ってください.
和文の組版の場合は,こうした単位の空け方が好まれます.

\item
二倍ダッシュの``\ddash ''は,
\verb/\ddash/ というマクロを使ってください(\ref{sec:typing} 参照).
``---'' を2つ重ねると,間に若干のスペースが入ることがあり
見苦しいからです.

\item
「証明終」を意味する記号 ``$\Box$'' を出力するマクロとして
\verb/\QED/ を定義しています\cite{tex}.
\verb/\hfill$\Box$/ では,この記号の直前の文字が行末に来る場合,
記号が行頭に来てしまいますので,\verb/\QED/ を使ってください.
$\Box$ を出力するには,{\ttfamily latexsym} パッケージが必要です.

\item
このクラスファイルでは,このほかに,
{\bfseries Table~\ref{table:1}} のマクロを定義しています.
\end{enumerate}


\begin{table}[t]% Table 3
\caption{Examples of macros defined in \JRSJcls{} file}
\label{table:1}
\begin{center}
\begin{tabular}{|c|c|}
\hline
macros & output \\
\hline
\verb/\RN{2}/   & \RN{2} \\
\verb/\RN{117}/ & \RN{117} \\
\verb/\FRAC{$\pi$}{2}/ & \FRAC{$\pi$}{2}\\
\verb/\FRAC{1}{4}/     & \FRAC{1}{4} \\
\verb/\MARU{1}/ & \MARU{1}\\
\verb/\MARU{a}/ & \MARU{a}\\
\verb/\kintou{4zw}{記号例}/       & \kintou{4zw}{記号例}\\
\verb/\ruby{砒}{ひ}\ruby{素}{そ}/ & \ruby{砒}{ひ}\ruby{素}{そ}\\
\hline
\end{tabular}
\end{center}
\end{table}


\subsection{AMS パッケージについて}

数式のより高度な記述のために,\AmSLaTeX{} の
パッケージ\cite{FMi1}を使う場合には,プリアンブルで
\begin{verbatim}
\usepackage{amsmath}
\end{verbatim}
と指定する必要があります.

{\ttfamily amsmath} パッケージは,多くの機能を提供していますが,
フォントとしてボールドイタリックだけを使いたい場合は,
\begin{verbatim}
\usepackage{amsbsy}
\end{verbatim}
で済みます.

また,記号類だけを使いたい場合は,
\begin{verbatim}
\usepackage{amssymb}
\end{verbatim}
で済みます.

なお,\LaTeXe{}では \verb/\mbox{\boldmath $x$}/ に代えて,
\verb/\boldsymbol{x}/ を使うことを勧めます.
数式の上付き・下付きで使うと文字が小さくなります.

\section{タイピングの注意事項}

\subsection{一般的な注意点}\label{sec:typing}

\begin{enumerate}
\item
和文の句読点は,``\makebox[1zw][c]{,}'' ``\makebox[1zw][c]{.}''
(全角記号)を使用してください.
和文中では,欧文用のピリオドとカンマ,``,'' ``.''(半角)は
使わないでください.

\item
括弧類についても,和文中で欧文を括弧でくくる場合は
全角の括弧を使用してください.
欧文中ではすべて半角ものを使用してください.

\noindent
例:スタイル(Style)ファイル / some (Style) files

上の例にように括弧のベースラインが異なります.

\item
ハイフン({\ttfamily -}),二分ダッシュ({\ttfamily --}),
全角ダッシュ({\ttfamily ---}),二倍ダッシュ(\verb/\ddash/)の
区別をしてください.

ハイフンはwell-knownなど一般的な欧単語の連結に,
二分ダッシュはpp.298--301のように範囲を示すときに,
全角ダッシュは欧文用連結のem-dash(---)として,
二倍ダッシュは(\ddash{})和文用連結として
使用してください.

\item
アラインメント以外の場所で,空行を広くとるため,\verb/\\/ による
強制改行を乱用するのはよくありません.

空行の直前に \verb/\\/ を入れたり,
\verb/\\/ を2つ重ねれば,確かに縦方向のスペースが広がりますが,
\verb/Underfull \hbox/ のメッセージがたくさん出力されて,
重要なメッセージを見落としがちになります\cite{jiyuu}.

\item
\verb*/( word )/ のように``( )''内や``( )''内の単語の前後に
スペースを入れないでください.

%\item
%本誌は,アスキー版\TeX{}でコンパイルして最終版下を作成するため,
%ギリシア文字の $\alpha$,$\beta$ などについては,その前後に
%``,'' ``.'' ``('' ``)''(全角JIS記号)がくる場合を除いて
%半角のスペースを入れるように心掛けてください.
%
%\leavevmode\phantom{$\Rightarrow$} 
%\verb*/ギリシア文字の$\alpha$,$\beta$については/\par
%$\Rightarrow$ ギリシア文字の$\alpha$,$\beta$については\par
%(アスキー版\TeX{}でコンパイルする場合,
%スペースを入れないとギリシア文字と全角文字の間がつまります)\par
%\leavevmode\phantom{$\Rightarrow$}
% \verb*/ギリシア文字の $\alpha$,$\beta$ については/\par
%$\Rightarrow$ ギリシア文字の $\alpha$,$\beta$ については

\item
プログラムリストなど,インデントが重要なものは,
力わざ(\verb/\hspace*{??mm}/ の使用や \verb/\\/ などによる強制改行)で
整形するのではなく,{\ttfamily list} 環境や {\ttfamily tabbing} 環境などを
使って赤字が入っても修正がしやすいように記述してください.
\end{enumerate}

\subsection{数式記述の注意点}\label{sec:eq1}

\begin{enumerate}
\item
数式モードの中でのハイフン,二分ダッシュ,マイナスの区別をしてください.\par
例えば,\par
\verb/$A^{\mathrm{b}\mbox{\tiny -}\mathrm{c}}$/\par
\hspace{2zw}$A^{\mathrm{b}\mbox{\tiny -}\mathrm{c}}$
 $\Rightarrow$ ハイフン\par
\verb/$A^{\mathrm{b}\mbox{\tiny --}\mathrm{c}}$/\par
\hspace{2zw}$A^{\mathrm{b}\mbox{\tiny --}\mathrm{c}}$
 $\Rightarrow$ 二分ダッシュ\par
\verb/$A^{b-c}$/\par
\hspace{2zw}$A^{b-c}$ $\Rightarrow$ マイナス\par
となります.それぞれの違いを確認してください.\par

\item
数式の中で,\verb/<,>/ を括弧のように使用することがよくみられますが,
数式中ではこの記号は不等号記号として扱われ,その前後にスペースが入ります.
このような形の記号を括弧として使いたいときは,
\verb/\langle/($\langle$),\verb/\rangle/($\rangle$)を
使うようにしてください.

\item
複数行の数式でアラインメントをするときに
数式が $+$ または $-$ で始まる場合,$+$ や $-$ は単項演算子と
みなされます(つまり,「$+x$」と「$x+y$」の $+$ の前後のスペースは
変わります).したがって,複数行の数式で $+$ や $-$ が先頭にくる場合は,
それらが2項演算子であることを示す必要があります\cite{latex}.
\begin{verbatim}
\begin{eqnarray}
y &=& a + b + c + ... + e\\
  & & \mbox{} + f + ... 
\end{eqnarray}
\end{verbatim}

\item
\TeX{}は,段落中の数式の中(\verb/$...$/)では改行を
うまくやってくれないことがあるので,その場合には \verb/\allowbreak/ を
使用することを勧めます\cite{Abrahams}.
\end{enumerate}

\subsection{長い数式を処理するには}\label{sec:eq2}

数式と数式番号が重なったり数式がはみ出したりする場合の
対処策を,いくつか挙げます.

\noindent
{\bfseries 例1}\hskip1zw \verb/\!/で縮める
\begin{eqnarray}
 y &=& a+b+c+d+e+f+g+h+i+j+k+l+m
\end{eqnarray}
のように数式と数式番号が重なるか,かなり接近する場合は,
2項演算記号や関係記号の前後を \verb/\!/ で
はさんで縮める方法があります.
\begin{verbatim}
\begin{equation}
 y &\!=\!& a \!+\! b \!+\! c \!+\! ... \!+\! m
\end{equation}
\end{verbatim}

縮めても,重なったりはみ出してしまう場合は,
\begin{verbatim}
\begin{eqnarray}
 y &=& a+b+c+d+e+f+g+h+i\nonumber\\
   & & \mbox{}+j+k+l+m 
\end{eqnarray}
\end{verbatim}
と記述すれば,
\begin{eqnarray}
 y &=& a+b+c+d+e+f+g+h+i\nonumber\\
   & & \mbox{}+j+k+l+m 
\end{eqnarray}
となります.

\noindent
{\bf 例2}\hskip1zw \verb/\lefteqn/ を使う
\begin{equation}
  \int\!\!\!\int_S \left({\partial V \over \partial x}
 -{\partial U \over \partial y}\right)dxdy=
 \oint_C \left(U{dx \over ds}+V{dy \over ds}\right)ds 
\end{equation}
\hskip1zw
上のように,$=$ までが長くて,数式がはみ出したり,
数式と数式番号が重なる場合には
(この例では離れていますが,重なっていると想定して)
\begin{verbatim}
\begin{eqnarray}
 \lefteqn{
  \int\!\!\!\int_S
   \left(\frac{\partial V}{\partial x}
    -\frac{\partial U}{\partial y}
   \right) dxdy
 }\quad \nonumber\\
 &=& \oint_C \left(U{dx \over ds}
      + V{dy \over ds}\right)ds
\end{eqnarray}
\end{verbatim}
と記述すれば,
\begin{eqnarray}
\lefteqn{ \int\!\!\!\int_S 
 \left(\frac{\partial V}{\partial x}
 -\frac{\partial U}{\partial y}\right)
 dxdy} \quad\nonumber\\
 &=& \oint_C \left(U{dx \over ds}
      + V{dy \over ds}\right)ds
\end{eqnarray}
となります.

\noindent
{\bf 例3}\hskip1zw パラメータを変える
\begin{equation}
A = \left(
  \begin{array}{cccc}
   a_{11} & a_{12} & \ldots & a_{1n} \\
   a_{21} & a_{22} & \ldots & a_{2n} \\
   \vdots & \vdots & \ddots & \vdots \\
   a_{m1} & a_{m2} & \ldots & a_{mn} \\
  \end{array}
    \right) \label{eq:ex1}
\end{equation}
\hskip1zw
上の行列では説明のために便宜上 {\ttfamily array} 環境を
使って記述しましたが,{\ttfamily array} 環境を使っていて,
数式がはみ出す場合は,
\begin{verbatim}
\begin{equation}
\arraycolsep=3pt %               <--- [1]
A = \left(
  \begin{array}{@{\hskip2pt}cccc@{\hskip2pt}}
%                   ↑ [2] 
   a_{11} & a_{12} & \ldots & a_{1n} \\
   a_{21} & a_{22} & \ldots & a_{2n} \\
   \vdots & \vdots & \ddots & \vdots \\
   a_{m1} & a_{m2} & \ldots & a_{mn} \\
  \end{array}
    \right) 
\end{equation}
\end{verbatim}
{\ttfamily [1]} のように,\verb/\arraycolsep/ の値(デフォルトは5\,pt)を
小さくしてみるか,{\ttfamily [2]} のように {\ttfamily @} 表現を
使うことができます.
\begin{equation}
\arraycolsep=3pt
A = \left(
  \begin{array}{@{\hskip2pt}cccc@{\hskip2pt}}
   a_{11} & a_{12} & \ldots & a_{1n} \\
   a_{21} & a_{22} & \ldots & a_{2n} \\
   \vdots & \vdots & \ddots & \vdots \\
   a_{m1} & a_{m2} & \ldots & a_{mn} \\
  \end{array}
    \right) \label{eq:ex2}
\end{equation}
式(\ref{eq:ex1})と式(\ref{eq:ex2})を比べてください.

\noindent
{\bf 例4}\hskip1zw 定義を変える

\makeatletter\ifx\@mathmargin\undefined\makeatother
行列を記述する場合に使用する \verb/\matrix/,
\verb/\pmatrix/ はコラムの間に \verb/\quad/ が挿入されているので,
間隔を縮めるには,ディスプレー数式環境の中で,
\verb/\def\quad/ の定義を変えてみてください.例えば,
\begin{equation}
 A = \pmatrix{
      a_{11} & a_{12} & \ldots & a_{1n} \cr
      a_{21} & a_{22} & \ldots & a_{2n} \cr
      \vdots & \vdots & \ddots & \vdots \cr
      a_{m1} & a_{m2} & \ldots & a_{mn} \cr
     }
\end{equation}
のような \verb/\pmatrix/ で記述した行列式で,
\verb/\quad/ の定義を変更すると
\begin{verbatim}
\begin{equation}
 \def\quad{\hskip.75em\relax}
 %% デフォルトは \hskip1em
 A = \pmatrix{
      a_{11} & a_{12} & \ldots & a_{1n} \cr
      a_{21} & a_{22} & \ldots & a_{2n} \cr

      \vdots & \vdots & \ddots & \vdots \cr
      a_{m1} & a_{m2} & \ldots & a_{mn} \cr
     }
\end{equation}
\end{verbatim}
\begin{equation}
 \def\quad{\hskip.75em\relax}
 A = \pmatrix{
      a_{11} & a_{12} & \ldots & a_{1n} \cr
      a_{21} & a_{22} & \ldots & a_{2n} \cr
      \vdots & \vdots & \ddots & \vdots \cr
      a_{m1} & a_{m2} & \ldots & a_{mn} \cr
     }
\end{equation}
となります.

\else
{\tt [pbvV]matrix} 環境などを使って
行列を記述する場合は,
\verb/\arraycolsep/ の値を変更します.
\begin{verbatim}
\begin{equation}
%% デフォルトは 5pt
 \arraycolsep3pt
 A =
  \begin{pmatrix}
   a_{11} & a_{12} & \ldots & a_{1n}\\
   a_{21} & a_{22} & \ldots & a_{2n}\\
   \vdots & \vdots & \ddots & \vdots\\
   a_{m1} & a_{m2} & \ldots & a_{mn}
  \end{pmatrix}
\end{equation}
\end{verbatim}

\begin{equation}
 \arraycolsep3pt
 A = \begin{pmatrix}
      a_{11} & a_{12} & \ldots & a_{1n} \\
      a_{21} & a_{22} & \ldots & a_{2n} \\
      \vdots & \vdots & \ddots & \vdots \\
      a_{m1} & a_{m2} & \ldots & a_{mn} 
     \end{pmatrix}
\end{equation}

\fi

以上に挙げたような処理でも数式がはみ出す場合には,
ディスプレー数式環境全体を {\ttfamily small},
{\ttfamily footnotesize} などで囲むことが考えられます.

\begin{thebibliography}{99}
\bibitem{tex}
 D.E. クヌース:\TeX{}ブック.
 アスキー出版局,1989.
\bibitem{ohno}
 大野義夫 編:\TeX{}入門.
 共立出版,1989.
\bibitem{Seroul}
 R. Seroul and S. Levy: A Beginner's Book of \TeX{}. 
 Springer-Verlag, 1989.
\bibitem{latex}
 レスリー・ランポート:文書処理システム\LaTeX{}. 
 アスキー出版局,1990.
\bibitem{nodera1}
 野寺隆志:楽々\LaTeX{}. 
 共立出版,1990.
\bibitem{bibunsho}
 奥村晴彦:\LaTeX{}美文書作成入門.
 技術評論社,1991.
\bibitem{itou}
 伊藤和人:\LaTeX{}トータルガイド.
 秀和システムトレーディング,1991.
\bibitem{nodera2}
 野寺隆志:今度こそ\AmSLaTeX{}. 
 共立出版,1991.
\bibitem{jiyuu}
 磯崎秀樹:\LaTeX{}自由自在.
 サイエンス社,1992.
\bibitem{impress}
 鷺谷好輝:日本語 \LaTeX{} 定番スタイル集.
 nos.1--3, インプレス,1992--1994.
\bibitem{Bech}
 S. von Bechtolsheim: \TeX{} in Practice. 
 vols.\RN{1}--\RN{4}, Springer-Verlag, 1993.
\bibitem{Gr}
 G. Gr\"{a}tzer:
 Math into \TeX{}\,--\,A Simple Introduction to \AmSLaTeX{}. 
 Birkh\"{a}user, 1993.
\bibitem{Kopka}
 H. Kopka and P.W. Daly: A Guide to \LaTeX{}. 
 Addison-Wesley, 1993.
\bibitem{hujita}
 藤田眞作:化学者・生化学者のための\LaTeX{}---パソコンによる論文作成の手引.
 東京化学同人,1993.
% \bibitem{styleuse}
% 古川徹生,岩熊哲夫:``\LaTeX{} のマクロやスタイルファイルの利用''
% (styleuse.tex),1994.
\bibitem{bibunsho2}
 奥村晴彦 監修:\LaTeX{} 入門---美文書作成のポイント.
 技術評論社,1994.
\bibitem{Ase}
 阿瀬はる美:てくてく\TeX{}.
 アスキー出版局,1994.
\bibitem{hujita2}
 藤田眞作:\LaTeX{} マクロの八衢.
 アジソン・ウェスレイ・パブリッシャーズ・ジャパン,1995.
\bibitem{Takayama}
 高山健三:``Inside DVI$\rightarrow$PS,'' 
 UNIX MAGAZINE, 1994--1996.
% \bibitem{Eijkhout}
%  V. Eijkhout: \TeX{} by Topic, Addison-Wesley, Wokingham, 1991.
\bibitem{Walsh}
 N. Walsh: Making \TeX{} Work. 
 O'Reilly \& Associates, 1994.
\bibitem{Salomon}
 D. Salomon: The Advanced \TeX{}book. 
 Springer-Verlag, 1995.
\bibitem{Nakano}
 中野賢:日本語 \LaTeXe{} ブック.
 アスキー出版局,1996.
\bibitem{Fujita4}
 藤田眞作:\LaTeXe{} 階梯.
 アジソン・ウェスレイ・パブリッシャーズ・ジャパン,1996.
\bibitem{}
 乙部巌己,江口庄英:p\LaTeXe{} for Windows\ \ Another Manual. 
 vols.0--2, ソフトバンク,1996--1997.
% \bibitem{FMi}
% M. Goossens, F. Mittelbach, and A. Samarin, ``The \LaTeX{} Companion,'' 
% Addison-Wesley, Reading, 1994.
\bibitem{Okumura3}
 奥村晴彦:\LaTeXe{} 美文書作成入門.
 技術評論社,1997.
% \bibitem{Abrahams}
%  P.W. Abrahams: \TeX{} for the Impatient,
%  Addison-Wesley, 1992.
\bibitem{Abrahams}
 ポール・W・エイブラハム:明快 \TeX{}.
 アジソン・ウェスレイ・パブリッシャーズ・ジャパン,1997.
\bibitem{Eguchi}
 江口庄英:Ghostscript Another Manual. 
 ソフトバンク,1997.
\bibitem{FMi2}
 M. Goossens, S. Rahts, and  F. Mittelbach: 
 The \LaTeX{} Graphics Companion. Addison-Wesley, 1997.
\bibitem{FMi1}
 マイケル・グーセンス,フランク・ミッテルバッハ,アレキサンダー・サマリン:
 \LaTeX{} コンパニオン.
 アスキー出版局,1998.
\bibitem{Eijkhout}
 ビクター・エイコー:\TeX{} by Topic---\TeX{}をよく深く知るための39章.
 アスキー出版局,1999.
\bibitem{FMi3}
 M. Goossens, and S. Rahts: 
 The \LaTeX{} Web Companion. 
 Addison-Wesley, 1999.
\end{thebibliography}

\appendix
\section{{\ttfamily jrsj-[rk].sty} との違い}
\label{sec:app}

\LaTeX~2.09 ベースの {\ttfamily jrsj-[rk].sty Ver.1} とは
まったく互換性がありません.記述方法が大幅に異なっているので
注意してください.

\section{標準のクラスファイルから削除したコマンド}

本誌の体裁に必要のないコマンドは削除しています.
削除したコマンドは,\verb/\part/,\allowbreak
\verb/\theindex/,\allowbreak
\verb/\tableofcontents/,\allowbreak
\verb/\titlepage/,\allowbreak
ページスタイルを変更するオプション({\ttfamily headings},
{\ttfamily myheadings})などです.

% \newpage 

\begin{biography}
\profile{n}{佐藤基昭(Sato Motoaki)}{%
株式会社ウルス(旧称 SATO工房),
〒113-0034 文京区湯島 1--7--9 お茶の水ウチヤマビル 5 階
(TEL 03--5803--9803 / FAX 03--5803--9804)}
\end{biography}

\end{document}
                                                                                                                                                                                                                                                                                                                                                                                                                                                                                                                                                                                                                                                                                                                                                                                                                                                                                                                                                                                                                                                         